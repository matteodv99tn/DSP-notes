\chapter{Introduction}

	\textbf{06/10/2021}
	
	\paragraph{Energy and power signals} Given a signal $x(t)$ with $t\in [-T/2, T/2]$ or a discrete one $x(n)$ with $n \in [-N/2,N/2]$, we define the \de{energy} $E_T$ and the \de{power} $P_T$ as
	\begin{equation}
	\begin{split}		
		E_T & = \int_{-T/2}^{T/2} |x(t)|^2 \, dt = \sum_{n=-N/2}^{N/2} |x(n)|^2 \\
		P_T & = \frac 1 T\int_{-T/2}^{T/2} |x(t)|^2 \, dt = \frac 1 N \sum_{n=-N/2}^{N/2} |x(n)|^2
	\end{split}
	\end{equation}
	
	This measures are called \textit{energy} and \textit{power} because if we use some physical signals, the evaluation of the integral/sum gives an energy (like for the example the energy/power generated by a voltage source joined by a $1\Omega$ resistor). 
	
	If we push the limit of $T$ we get the \textbf{energy signal} $E = \lim_{T\rightarrow \infty} E_T < \infty$ and the \textbf{power signal} $P = \lim_{T\rightarrow \infty} P_T < \infty$ (note that this 2 values must be limited). This definitions allows us also to define the \de{scalar product between signals}, and in particular for the energy signal it evaluates
	\begin{equation}
		\langle x,y\rangle = \int_{-\infty}^\infty x(t) y^*(t)\, dt
	\end{equation}
	where $y^*$ is the conjugate of the signal $y$. For the power signal instead the definition of the scalar product is described as
	\begin{equation}
		\langle x, y\rangle := \lim_{\Delta t \rightarrow \infty} \frac{1}{\Delta t} \int_{-\Delta t/2}^{\Delta t/2} x(t)y^*(t) \, dt
	\end{equation}
	We can also notice that, for a periodic signal with period $T$ we can consider it's power as $ \frac{1}{T} \int_{-T/2}^{T/2} x(t)y^*(t) \, dt$. And another fact is that the energy of the signal $x(t)$ is defined as $E= \langle x,x \rangle$.
	